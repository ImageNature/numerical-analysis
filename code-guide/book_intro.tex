%%%%%%%%%%%%%%%%%%%%%%%%%%%%%%%%%%%%%%%%%%%%%%%%%%%%%%%%%%%%%%%%%%%%%%%%%%%%%%%
%%%
%%%   VOORWOORD
%%%
%%%%%%%%%%%%%%%%%%%%%%%%%%%%%%%%%%%%%%%%%%%%%%%%%%%%%%%%%%%%%%%%%%%%%%%%%%%%%


\chapter{前言}
\label{cha:intro}
\thispagestyle{empty}

Een van de beste boeken over C is \textsl{The C Programming Language} van
Brian Kernighan en Dennis Ritchie~\cite{kernighan1988c}.

\begin{displayquote}
The book is not an introductory programming manual; it assumes some familiarity
with basic programming concepts like variables, assignment statements,
loops, and functions.
[...]
C is not a big language, and it is not well served by a big book.
\end{displayquote}

.

Dit boek is opgemaakt in \LaTeX~\cite{latexwebsite} (\LaTeX-engine = \booktexbanner).
\LaTeX\@ leent zich uitstekend
voor het opmaken van lopende tekst, tabellen, figuren, programmacode, vergelijkingen
en referenties. De gebruikte \LaTeX-distributie is TexLive uit 2021~\cite{texlivewebsite}.
Als editor is TexStudio~\cite{texstudiowebsite} gebruikt. Tekst is gezet in
\ifnum 0\ifxetex 1\fi\ifluatex 1\fi>0
Calibri~\cite{calibrifont}, een van de standaard fonts op een bekend besturingssysteem.
Code is opgemaakt in Consolas~\cite{consolasfont}
\else
Charter~\cite{charterfont}, een van de standaard fonts in \LaTeX.
De keuze hiervoor is dat het een prettig te lezen lettertype is, een
\textsl{slanted} letterserie heeft en een bijbehorende wiskundige tekenset heeft.
Code is opgemaakt in Nimbus Mono~\cite{nimbusfont}
\fi
met behulp van de \textsl{listings}-package~\cite{listingsctan}.
Voor het
tekenen van arrays, pointers en flowcharts is \textsl{TikZ/PGF} gebruikt~\cite{tikzctan}.
Alle figuren zijn door de auteur zelf ontwikkeld, behalve de logo's van Creative
Commons en De Haagse Hogeschool.
Een aantal programmafragmenten is ontwikkeld door collega Harry Broeders, de overige
fragmenten zijn door de auteur ontwikkeld.

van dit boek is beschikbaar op \url{https://github.com/jesseopdenbrouw/book_c}. We doen
een klemmend beroep aan alle lezers om fouten en omissies te melden.




\bigskip
\hfill \author, \ifcase\month \or januari\or februari\or maart\or april\or mei\or juni\or juli\or augustus\or september\or oktober\or november\or december\fi\ \the\year.