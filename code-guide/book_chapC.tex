\chapter{Voorrangsregels van operatoren}
\label{cha:voorrang}\index{voorrangsregels}
\thispagestyle{empty}

In deze tabel zijn alle operatoren in C opgesomd. De voorrang of prioriteit is in aflopende volgorde van hoog naar laag. Twee opmerkelijke operatoren zijn (\textsl{type cast}) en \texttt{sizeof}. Bij (\textsl{type cast}) wordt het casten van een enkelvoudig datatype bedoeld en dat kan ook een pointer (naar een variabele) zijn, \texttt{sizeof} berekent de grootte in bytes van een datatype of variabele.

\begin{table}[!ht]
\centering
\renewcommand{\arraystretch}{1.2}
\caption{Voorrangsregels van alle operatoren.}
\label{tab:bijvoorrangsregels}
\begin{tabular}{p{9cm}l}
\toprule
\textbf{Operator} & \textbf{Associativiteit} \\
\midrule
\texttt{() [] -> .} & links naar rechts \\
\texttt{! \textasciitilde\ + - ++ -- (}\textsl{type cast}\texttt{)} \texttt{sizeof * \&} & rechts naar links \\
\texttt{* / \%} & links naar rechts \\
\texttt{+ -} & links naar rechts \\
\texttt{<< >>} & links naar rechts\\
\texttt{< <= > >=} & links naar rechts\\
\texttt{== !=} & links naar rechts\\
\texttt{\&} & links naar rechts\\
\texttt{\^{}} & links naar rechts\\
\texttt{\textbar} & links naar rechts\\
\texttt{\&\&} & links naar rechts\\
\texttt{\textbar\textbar} & links naar rechts\\
\texttt{?:} & rechts naar links \\
\texttt{= += -= *= /= \%= \&= \^{}= \textbar= <<= >>=} & rechts naar links \\
\texttt{,} & links naar rechts \\
\bottomrule
\end{tabular}
\end{table}

Let erop dat \texttt{sizeof} een compile-time\index{compile-time} operator is, maar kan vanaf C99 ook gebruikt worden bij een dynamisch gedefinieerde array, bijvoorbeeld:

\hspace*{1em}\texttt{int a[b];}

en \texttt{b} is niet bekend tijdens compile-time maar wel tijdens run-time.

%\begin{lstlisting}[caption=Gebruik van \texttt{sizeof} tijdens runtime.]
%#include <stdio.h>
%
%int some_function(int some_array[], int b) {
%
%    int a[b], i;
%    
%    for
%}
%\end{lstlisting}
